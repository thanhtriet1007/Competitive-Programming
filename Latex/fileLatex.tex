\documentclass[12pt]{article}
\usepackage{amsmath}
\usepackage{graphicx}
\usepackage{geometry}
\geometry{a4paper}
\usepackage[utf8]{inputenc} 
\usepackage[vietnamese]{babel} 
\usepackage{array} 

\geometry{top=1cm, bottom=1.5cm, left=1cm, right=1cm}
\begin{document}

\title{Đầu tư - CITY}
\author{Binh Thuan Summer Camp - VOI26 Training}
\maketitle

\setlength{\parindent}{0pt}
\setlength{\parskip}{1em} 

\subsection*{Mô tả bài toán}
Thành phố TrieCountry hiện đang có X + Y + Z mảnh đất trống, được đánh số từ 1 đến X + Y + Z.
Anh Long - một tỉ phú của thành phố TrieCountry, chủ tịch tập đoàn Real Betis - đã quyết định mua
lại X + Y + Z mảnh đất này. Với mảnh đất thứ $i$, anh có thể xây dựng một trong ba công trình
sau:

\begin{itemize}
  \item Trung tâm thương mại với lợi nhuận $A_i$.
  \item Khu vui chơi với lợi nhuận $B_i$.
  \item Nhà hàng với lợi nhuận $C_i$.
\end{itemize}

Qua nghiên cứu thị trường, anh Long cho rằng nên xây dựng đúng X trung tâm mua sắm, Y khu
vui chơi và Z nhà hàng. Hãy tính tổng lợi nhuận tối đa mà anh Long có thể thu được nếu xây
dựng các công trình trên một cách tối ưu.

\subsection*{Dữ liệu}
\begin{itemize}
  \item Dòng đầu tiên chứa ba số nguyên dương X, Y , Z $(X + Y + Z \leq 10^5)$ - số trung tâm thương mại, khu vui chơi và nhà hàng cần xây dựng.
  \item X + Y + Z dòng tiếp theo, mỗi dòng gồm ba số nguyên $A_i, B_i, C_i$ $(0 \leq A_i, B_i, C_i \leq 10^9)$ lần 
  lượt là lợi nhuận thu được khi xây dựng trung tâm thương mại, khu vui chơi và nhà hàng ở mảnh đất thứ i.
\end{itemize}

\subsection*{Kết quả}
\begin{itemize}
  \item Gồm một số nguyên duy nhất là tổng lợi nhuận tối đa.
\end{itemize}

\subsection*{Cách tính điểm} 
\begin{itemize}
  \item Subtask 1 (20\% số điểm): $B_i = C_i$.
  \item Subtask 2 (30\% số điểm): $C_i = 0$.
  \item Subtask 3 (50\% số điểm): Không có ràng buộc gì thêm.
\end{itemize}

\subsection*{Ví dụ}
\begin{center}
\begin{tabular}{|>{\raggedright\arraybackslash}p{8cm}|>{\raggedright\arraybackslash}p{8cm}|}
\hline
\textbf{input} & \textbf{output} \\
\hline
3 2 1 & 35 \\
0 3 2 & \\
1 4 9 & \\
5 3 2 & \\
7 5 9 & \\
4 8 9 & \\
3 0 4 & \\
      & \\
      & \\
\hline
\end{tabular}
\end{center}

\end{document}
