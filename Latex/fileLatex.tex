\documentclass[12pt]{article}
\usepackage{amsmath}
\usepackage{graphicx}
\usepackage{geometry}
\geometry{a4paper}
\usepackage[utf8]{inputenc} 
\usepackage[vietnamese]{babel} 
\usepackage{array} 

\geometry{top=1cm, bottom=1.5cm, left=1cm, right=1cm}
\begin{document}

\title{Hành trình leo núi - CLIMBING}
\author{Binh Thuan Summer Camp - VOI26 Training}
\maketitle

\setlength{\parindent}{0pt}
\setlength{\parskip}{1em} 

\subsection*{Mô tả bài toán}
Sau một thời gian lưu lạc xứ sở Shurima, Triết quyết định lên đường đi leo núi để thử thách bản thân. Địa điểm
Triết leo rất đặc biệt. Ở đây có $n$ đỉnh núi được đánh số từ $1$ tới $n$. Đỉnh núi thứ $i$ có độ cao là $a_i$.
Nhưng Triết lại khá lười nên việc leo cả $n$ ngọn núi là một điều vô cùng khó khăn với anh. Vì thế, Triết
đã nghĩ ra kế hoạch leo núi vô cùng thông minh và tiết kiệm công sức. Theo kế hoạch thì Triết sẽ
xuất phát tại một đỉnh núi nào đó và leo đúng $m$ ngọn núi (bao gồm cả đỉnh núi xuất phát). Đồng
thời để việc leo núi đỡ vất vả hơn thì chênh lệch độ cao giữa của đỉnh núi cao nhất và thấp nhất
trong kế hoạch của Triết không vượt quá $c$. Sau khi nghĩ ra kế hoạch siêu trí tuệ này thì Triết lại
không biết những đỉnh núi mình nên xuất phát là đỉnh núi nào vì vậy Triết nhờ các bạn hãy giúp
Triết nhé!

\subsection*{Dữ liệu}
\begin{itemize}
  \item Dòng đầu tiên gồm số 3 nguyên: $n$ là số lượng đỉnh núi $(1 \leq n \leq 1000000)$, $m$ là số lượng
  đỉnh núi liên tiếp cần leo $(1 \leq m \leq min(n, 10000))$, $c$ là chênh lệch tối đa $(0 \leq c \leq 10000)$.
  \item Dòng thứ hai gồm $n$ số nguyên $a_i$ là độ cao của núi thứ $i$ $(0 \leq a_i \leq 10000000)$.
\end{itemize}

\subsection*{Kết quả}
\begin{itemize}
  \item Liệt kê tất cả vị trí đỉnh núi khác nhau mà Triết có thể xuất phát để đạt được kế hoạch của mình.
  Các vị trí được in ra theo thứ tự tăng dần và mỗi dòng là một vị trí. Nếu không có vị trí nào thỏa
  thì in ra NONE.
\end{itemize}


\subsection*{Ví dụ}
\begin{center}
\begin{tabular}{|>{\raggedright\arraybackslash}p{8cm}|>{\raggedright\arraybackslash}p{8cm}|}
\hline
\textbf{input} & \textbf{output} \\
\hline
7 2 0 & 2 \\
0 1 1 2 3 2 2 & 6\\
& \\
& \\
& \\
& \\
& \\
& \\
& \\
\hline
\end{tabular}
\end{center}

\end{document}
