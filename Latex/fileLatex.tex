\documentclass[12pt]{article}
\usepackage{amsmath}
\usepackage{graphicx}
\usepackage{geometry}
\geometry{a4paper}
\usepackage[utf8]{inputenc} 
\usepackage[vietnamese]{babel} 
\usepackage{array} 

\geometry{top=1cm, bottom=1.5cm, left=1cm, right=1cm}
\begin{document}

\title{Chiều cao - HEIGHT}
\author{Binh Thuan Summer Camp - VOI26 Training}
\maketitle

\setlength{\parindent}{0pt}
\setlength{\parskip}{1em} 

\subsection*{Mô tả bài toán}
Trie đang trong tiết học thể dục. Thầy giáo bảo cả lớp xếp thành một hàng ngang. 
Lớp học của Trie có n học sinh, khi xếp thành hàng ngang, các học sinh được đánh số từ 1 tới n theo 
thứ tự từ trái qua phải. Học sinh thứ i có chiều cao $h_i$.

Hai học sinh i và j có thể nhìn thấy nhau nếu như giữa họ không có học sinh nào có chiều cao lớn hơn. 
Cụ thể hơn, học sinh i và j ($i < j$) nhìn thấy nhau nếu như $h_k \leq h_i$ và $h_k \leq h_j$ ($\forall i < k < j$).

Trie muốn biết với mỗi học sinh, người đó có thể nhìn thấy bao nhiêu học sinh khác mà có cùng chiều cao với họ.

\subsection*{Dữ liệu}
\begin{itemize}
  \item Dòng đầu tiên chứa số nguyên dương \(q \) (1 $\leq$ $q$ $\leq$ 10) - số truy vấn.
  \item Mỗi truy vấn gồm hai dòng, dòng thứ nhất chứa số nguyên dương $n$ (1 $\leq$ $n$ $\leq$ $10^5$).
  \item Dòng thứ hai chứa $n$ số nguyên dương $h_1, h_2, h_3,..., h_n $ (1 $\leq$ $h_i$ $\leq$ $10^9$).
\end{itemize}

\subsection*{Kết quả}
\begin{itemize} 
  \item Với mỗi truy vấn, in ra trên một dòng n số nguyên cách nhau bởi dấu cách là câu trả lời cho truy vấn đó.
\end{itemize}

\subsection*{Subtask}
\begin{itemize}
  \item Subtask 1: \(n\) không vượt quá 1000.
  \item Subtask 2: Không có ràng buộc gì thêm.
\end{itemize}

\subsection*{Ví dụ}

\begin{center}
\begin{tabular}{|>{\raggedright\arraybackslash}p{8cm}|>{\raggedright\arraybackslash}p{8cm}|}
\hline
\textbf{input} & \textbf{output} \\
\hline
1 & 0 1 1 0 0 \\
5&  \\
1 2 2 3 2 &  \\
\hline
\end{tabular}
\end{center}

\end{document}
