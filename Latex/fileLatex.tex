\documentclass[12pt]{article}
\usepackage{amsmath}
\usepackage{graphicx}
\usepackage{geometry}
\geometry{a4paper}
\usepackage[utf8]{inputenc} 
\usepackage[vietnamese]{babel} 
\usepackage{array} 

\geometry{top=1cm, bottom=1.5cm, left=1cm, right=1cm}
\begin{document}

\title{AQUERY}
\author{Binh Thuan Summer Camp - VOI26 Training}
\maketitle

\setlength{\parindent}{0pt}
\setlength{\parskip}{1em} 

\subsection*{Mô tả bài toán}
Cho một dãy A gồm N phần tử. Ban đầu, giá trị của các phần tử đều bằng 0. Có Q truy vấn,
truy vấn thứ i được mô tả bởi hai số nguyên $r_i$ và $p_i$, yêu cầu thực hiện $p_i$ lần các thao tác sau:
\begin{itemize}
  \item Chọn phần tử có giá trị nhỏ nhất trong các phần tử có vị trí từ 1 đến ri. Nếu có nhiều phần
  tử có cùng giá trị nhỏ nhất, chọn phần tử có vị trí nhỏ nhất trong số chúng.
  \item Tăng giá trị của phần tử được chọn thêm 1.
\end{itemize}
Hãy cho biết giá trị các phần tử trong dãy A sau khi thực hiện Q truy vấn trên.

\subsection*{Dữ liệu}
\begin{itemize}
  \item Dòng đầu tiên gồm hai số nguyên N, Q $(1 \leq N, Q \leq 10^5)$ - số phần tử của dãy A và số truy
  vấn cần thực hiện.
  \item Q dòng tiếp theo, mỗi dòng gồm hai số nguyên $r_i$ và $p_i$ $(1 \leq r_i \leq N, 1 \leq p_i \leq 9 * 10^8)$ - mô tả truy vấn thứ i.
\end{itemize}

\subsection*{Kết quả}
\begin{itemize}
  \item In ra N số nguyên lần lượt là giá trị các phần tử trong dãy A sau khi thực hiện Q truy vấn
\end{itemize}


\subsection*{Ví dụ}
\begin{center}
\begin{tabular}{|>{\raggedright\arraybackslash}p{8cm}|>{\raggedright\arraybackslash}p{8cm}|}
\hline
\textbf{input} & \textbf{output} \\
\hline
8 3  & 4 4 3 3 3 2 1 1 \\
3 11 &\\
8 7  & \\
6 3  & \\
& \\
& \\
& \\
& \\
& \\
\hline
\end{tabular}
\end{center}

\end{document}
