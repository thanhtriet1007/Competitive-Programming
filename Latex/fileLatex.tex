\documentclass[12pt]{article}
\usepackage{amsmath}
\usepackage{graphicx}
\usepackage{geometry}
\geometry{a4paper}
\usepackage[utf8]{inputenc} 
\usepackage[vietnamese]{babel} 
\usepackage{array} 

\geometry{top=1cm, bottom=1.5cm, left=1cm, right=1cm}
\begin{document}

\title{Trò chơi trên vòng tròn -- CIRCLE}
\author{Binh Thuan Summer Camp - VOI26 Training}
\maketitle

\subsection*{Mô tả bài toán}
Cho một tập số nguyên rỗng rạc, trong đó mỗi số nguyên trị từ 1 đến N thuộc vào một trong M tập con của tập số nguyên này. Các tập con 1, 2, 3, \dots, M được sắp xếp lần lượt trong tập các tập con theo chiều kim đồng hồ (tập M là tập cuối cùng).

\subsection*{Dữ liệu}
\begin{itemize}
  \item Dòng đầu tiên cho biết số nguyên dữ liệu N, M, Q.
  \item Dòng thứ 2 cho biết D chiều dài dữ liệu.
\end{itemize}

\subsection*{Kết quả}
Giới hạn số lượng tối đa các kết quả đáp trả.

\subsection*{Subtask}
\begin{itemize}
  \item Subtask 1: \( T \) không vượt quá 1,000.
  \item Subtask 2: \( T \) không vượt quá 100.
  \item Subtask 3: Số chấp nhận là.
\end{itemize}

\subsection*{Ví dụ}

\begin{center}
\begin{tabular}{|>{\raggedright\arraybackslash}p{8cm}|>{\raggedright\arraybackslash}p{8cm}|}
\hline
\textbf{input} & \textbf{output} \\
\hline
2 & 11+0+1101 \\
1101101 & 10+1+0+1 \\
10101 &  \\
\hline
\end{tabular}
\end{center}

\end{document}
