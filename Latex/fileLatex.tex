\documentclass[12pt]{article}
\usepackage{amsmath}
\usepackage{graphicx}
\usepackage{geometry}
\geometry{a4paper}
\usepackage[utf8]{inputenc} 
\usepackage[vietnamese]{babel} 
\usepackage{array} 

\geometry{top=1cm, bottom=1.5cm, left=1cm, right=1cm}
\begin{document}

\title{Du lịch - TRAVEL}
\author{Binh Thuan Summer Camp - VOI26 Training}
\maketitle

\setlength{\parindent}{0pt}
\setlength{\parskip}{1em} 

\subsection*{Mô tả bài toán}
Đất nước TrieCountry có $N$ thành phố và $N - 1$ cung đường nối các thành phố với nhau. Có thể đi từ thành phố này qua thành phố khác bằng cách đi ngang qua các cung đường có sẵn. Thành phố i có giá trị du lịch là $a_i$. Triết là một người đam mê khám phá, nên cậu ta quyết định đi du lịch đến đất nước TrieCountry. Chi phí để đi qua cung đường nối 2 thành phố i và j là $ \max(a_i + a_j, |a_i - a_j|)$. Triết sẽ du lịch ở thành phố $Q$ ngày, tại một ngày bất kì, chính phủ sẽ thay đổi giá trị du dịch
của thành phố  $u$ thành một giá trị mới là $X$, ngày đó du lịch sẽ không hoạt động và Triết sẽ không
đi được đâu. Do để được tham quan nhiều nhất có thể, mỗi ngày ở quốc gia TrieCountry có thể tham
quan Triết sẽ xuất phát từ thành phố $u$ và đến thành phố $v$. Để tiết kiệm chi phí Minh sẽ chọn
lộ trình đi có giá thành rẻ nhất.

Với $Q$ ngày ở đất nước TrieCountry, các bạn hãy tính giúp Triết chi phí du lịch nếu như ngày đó thành
phố không dừng hoạt động du lịch.

\subsection*{Dữ liệu}
\begin{itemize}
  \item Dòng đầu tiên chứa hai số nguyên $N, Q$ $(1 \leq N \leq 10^5, 1 \leq Q \leq 10^5)$.
  \item Dòng tiếp theo chứa $N$ số nguyên. Số nguyên thứ i là $a_i (-10^9 \leq a_i \leq 10^9)$.
  \item $N - 1$ dòng tiếp theo chứa 2 số nguyên u và v khác nhau biểu thị cho cung đường nối hai thành phố u và v $(1 \leq u, v \leq N)$.
  \item $Q$ dòng tiếp theo, chứa 3 số nguyên $q, u, v$. Nếu $q = 1$ biểu thị cho ngày hôm đó chính phủ sẽ dừng hoạt động du lịch và thay đổi giá trị du lịch của thành phố u thành v. Nếu $q = 2$ biểu thị cho ngày đó Triết sẽ đi du lịch từ thành phố u đến thành phố v.
\end{itemize}

\subsection*{Kết quả}
\begin{itemize}
  \item Với mỗi ngày Triết được du lịch, hãy xuất ra chi phí du lịch của ngày hôm đó.
\end{itemize}

\subsection*{Ví dụ}
\begin{center}
\begin{tabular}{|>{\raggedright\arraybackslash}p{8cm}|>{\raggedright\arraybackslash}p{8cm}|}
\hline
\textbf{input} & \textbf{output} \\
\hline
4 4 & 2102 \\
2 -1000 100 3 & 0 \\
2 1 & 1000000003 \\
3 2 &  \\
4 1 & \\
2 1 3 & \\
2 2 2 & \\
1 1 -1000000000 & \\
2 1 4 & \\
\hline
\end{tabular}
\end{center}

\end{document}
