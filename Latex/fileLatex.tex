\documentclass[12pt]{article}
\usepackage{amsmath}
\usepackage{graphicx}
\usepackage{geometry}
\geometry{a4paper}
\usepackage[utf8]{inputenc} 
\usepackage[vietnamese]{babel} 
\usepackage{array} 

\geometry{top=1cm, bottom=1.5cm, left=1cm, right=1cm}
\begin{document}

\title{Cửa sổ - SURAJ}
\author{Binh Thuan Summer Camp - VOI26 Training}
\maketitle

\setlength{\parindent}{0pt}
\setlength{\parskip}{1em} 

\subsection*{Mô tả bài toán}
Một coder trẻ với mật danh Hecker đã chiến thắng giải đấu Surajet Open và giành được chiếc laptop Surajook độc quyền của nhà tài trợ Suraj. Chiếc laptop này có cài đặt sẵn một trình duyệt internet do công ty Suraj phát triển vói các thử nghiệm hết sức độc đáo. Tuy nhiên trình duyệt bị giới hạn khi chỉ có thể mở được nhiều nhất k cửa sổ và tab thứ i trong mỗi cửa sổ i chiếm i megabytes trong bộ nhớ. Hecker cho biết chiếc laptop mới có bộ nhớ m megabytes. Bạn hãy giúp Hecker tính xem câu ta có thể mở được nhiều nhất là bao nhiêu tab.

\subsection*{Dữ liệu}
\begin{itemize}
  \item Dòng đầu tiên gồm một số nguyên $T$ $(1 \leq T \leq 10^5)$.
  \item Dòng thứ i + 1 $(1 \leq i \leq T)$ gồm hai số nguyên $m$, $k$ $(1 \leq m \leq 10^{18}$, $1 \leq k \leq 10^9)$ là tổng bộ nhớ của Surajook và số cửa sổ nhiều nhất có thể mở.
\end{itemize}

\subsection*{Kết quả}
\begin{itemize}
  \item Gồm $T$ dòng, mỗi dòng gồm một số nguyên duy nhất là số tab nhiều nhất mà Hecker có thể mở.
\end{itemize}

\subsection*{Ví dụ}
\begin{center}
\begin{tabular}{|>{\raggedright\arraybackslash}p{8cm}|>{\raggedright\arraybackslash}p{8cm}|}
\hline
\textbf{input} & \textbf{output} \\
\hline
2 & 10 \\
23 3 & 2 \\
2 3 & \\
\hline
\end{tabular}
\end{center}

\end{document}
